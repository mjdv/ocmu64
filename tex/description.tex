\documentclass[a4paper,UKenglish,cleveref, autoref, thm-restate]{lipics-v2021}
%for section-numbered lemmas etc., use "numberwithinsect"
%for enabling thm-restate support, use "thm-restate"

\pdfoutput=1 %uncomment to ensure pdflatex processing (mandatatory e.g. to submit to arXiv)
\hideLIPIcs  %uncomment to remove references to LIPIcs series (logo, DOI, ...), e.g. when preparing a pre-final version to be uploaded to arXiv or another public repository

\graphicspath{{./fig/}}%helpful if your graphic files are in another directory

\bibliographystyle{plainurl}% the mandatory bibstyle

\title{OCMu64: a solver for One-sided Crossing Minimization} %TODO Please add

%\titlerunning{Dummy short title} %TODO optional, please use if title is longer than one line

\author{Ragnar {Groot Koerkamp}}{ETH Zurich, Switzerland}{ragnar.grootkoerkamp@gmail.com}{https://orcid.org/0000-0002-2091-1237}{}%TODO mandatory, please use full name; only 1 author per \author macro; first two parameters are mandatory, other parameters can be empty. Please provide at least the name of the affiliation and the country. The full address is optional. Use additional curly braces to indicate the correct name splitting when the last name consists of multiple name parts.

\author{Mees de Vries}{Unaffiliated, Netherlands}{meesdevries@protonmail.com}{}{}

\authorrunning{R. Groot Koerkamp and M. de Vries} %TODO mandatory. First: Use abbreviated first/middle names. Second (only in severe cases): Use first author plus 'et al.'

\Copyright{Ragnar Groot Koerkamp and Mees de Vries} %TODO mandatory, please use full first names. LIPIcs license is "CC-BY";  http://creativecommons.org/licenses/by/3.0/

\ccsdesc[500]{Theory of computation~Mathematical optimization}
\ccsdesc[300]{Theory of computation~Computational geometry}

\keywords{Graph drawing, crossing number, branch and bound} %TODO mandatory; please add comma-separated list of keywords

\category{} %optional, e.g. invited paper

\relatedversion{\url{https://doi.org/10.5281/zenodo.11671980}} %optional, e.g. full version hosted on arXiv, HAL, or other respository/website
%\relatedversiondetails[linktext={opt. text shown instead of the URL}, cite=DBLP:books/mk/GrayR93]{Classification (e.g. Full Version, Extended Version, Previous Version}{URL to related version} %linktext and cite are optional

\supplement{}%optional, e.g. related research data, source code, ... hosted on a repository like zenodo, figshare, GitHub, ...
\supplementdetails[subcategory={Source Code}]{Software}{https://github.com/mjdv/ocmu64} %linktext, cite, and subcategory are optional

%\funding{(Optional) general funding statement \dots}%optional, to capture a funding statement, which applies to all authors. Please enter author specific funding statements as fifth argument of the \author macro.

%% \acknowledgements{I want to thank \dots}%optional

\nolinenumbers %uncomment to disable line numbering



%Editor-only macros:: begin (do not touch as author)%%%%%%%%%%%%%%%%%%%%%%%%%%%%%%%%%%
%% \EventEditors{John Q. Open and Joan R. Access}
%% \EventNoEds{2}
%% \EventLongTitle{42nd Conference on Very Important Topics (CVIT 2016)}
%% \EventShortTitle{CVIT 2016}
%% \EventAcronym{CVIT}
%% \EventYear{2016}
%% \EventDate{December 24--27, 2016}
%% \EventLocation{Little Whinging, United Kingdom}
%% \EventLogo{}
%% \SeriesVolume{42}
%% \ArticleNo{23}
%%%%%%%%%%%%%%%%%%%%%%%%%%%%%%%%%%%%%%%%%%%%%%%%%%%%%%

\newcommand{\R}{\mathbb R}
\renewcommand{\b}{\prec}
\newcommand{\be}{\preceq}
\newcommand{\g}{\sim}
\renewcommand{\u}{\overline{u}}
\renewcommand{\v}{\overline{v}}
\newcommand{\w}{\overline{w}}
\DeclareMathOperator{\supp}{supp}

\begin{document}

\maketitle

%TODO mandatory: add short abstract of the document
\begin{abstract}
  Given a bipartite graph $(A,B)$, the \emph{one-sided crossing minimization} (OCM) problem is to find an order of the
  vertices of $B$ that minimizes the number
  of edge crossings when drawn in the plane.

  We prove a number of novel reductions, and apply these in our solver OCMu64 that
  uses branch-and-bound on the order of the vertices of $B$.
\end{abstract}

\section{Introduction}

The 2024 edition of PACE, an annual optimization challenge, considers the
\emph{one-sided crossing minimization} problem, defined as follows.
Given is a bipartite graph $(A, B)$ that is drawn in the plane at points
$(i, 1)$ and $(j,0)$ for components $A$ and $B$ respectively. The ordering of $A$
is fixed, and the goal is the find an ordering of $B$ that minimizes the number of
crossings when edges are drawn as straight lines.

We introduce some new reductions and give an overview of our algorithm. Proofs are brief or
omitted due to lack of space.

\section{Definitions}
We use $<$ to compare vertices in $A$ in their fixed ordering. We generalize to weighted graphs
for the proof of \cref{lem:sfp}: a node $u \in B$ is taken to be a function $u: A \to \mathbb
R^{\geq 0}$, where weight 0 is an absent edge. We write $W_u = \sum_{a \in A}u(a)$ for the
total weight of $u$, and set $\bar u = u/W_u$. Write $N(u) = \supp(u) \subseteq A$ for the
set of neighbors of $u$.

We write $c(u, v) = \sum_{(a, b) \in A^2}
u(a)v(b)[a > b]$; in the unweighted case, this is the \emph{crossing number}, the number of
crossings between edges incident to $u$ and $v$ when $u$ is drawn before $v$.  For
$X,Y\subseteq B$ we set $c(X,Y) = \sum_{x\in X}\sum_{y\in Y} c(x,y)$ for the cost of ordering
all vertices of $X$ before all vertices of $Y$. More generally, $c(X,Y,Z) = c(X, Y) + c(X, Z) +
c(Y, Z)$. We also consider the \emph{reduced cost} $r(X,Y) = c(X, Y) - c(Y, X)$, which is
negative when $X$ is better \emph{before} $Y$ and positive when $X$ is better \emph{after} $Y$.

We write $u\b v$ when $u$ must come before $v$ in all minimal solutions, and say that $(u, v)$
is a \emph{fixed} pair. When there exists a minimal solution where $u$ comes before $v$, we
write $u \be v$ and call $(u, v)$ \emph{weakly fixed}.
We write $u\g v$ when there exists a minimal solution where $u$ and $v$ are
consecutive, in that order.

The following result is well known.

% TODO: Citation
\begin{observation}\label{lem:weak}
  When all neighbours of $u$ come before all neighbours of $v$, then $u$ goes before $v$. That
    is, when $\max N(u) \leq \min N(v)$, then $u\be v$, and in fact $u \b v$ holds
    unless $u$ and $v$ are both connected to only one and the same node in $A$.
\end{observation}

\section{Methods}
\subsection{Reductions}
\subparagraph{Fixed pairs}
We give a much stronger version of \cref{lem:weak}. In fact, this
lemma is as strong as can be when only considering $u, v$ themselves. We begin with this
abstract version, followed by a concrete corrolary.

\begin{definition}
    We call $u, v \in B$ a \emph{strongly fixed pair} if for every $b \in A$ we have ${\sum_{a
    \leq b} \bar u(a) \geq \sum_{a \leq b} \bar v(a)}$, and at least one of these inequalities
    is strict. This implies $r(u, v) {<} 0$.
\end{definition}
\begin{lemma}[Strongly fixed pair]\label{lem:sfp} When $u, v$ is a strongly fixed pair,
    then for any $w: A\to \mathbb R^{\geq 0}$ we have $r(w, \bar v) \leq r(w, \bar u)$.
\end{lemma}
\begin{proof}
    Consider the least element $a_0 \in A$ such that $\bar u(a_0) \neq
    \bar v(a_0)$. We must have $\bar u(a_0) > \bar v(a_0)$. Now consider the transformation of
    $\bar u$ which ``moves'' $\delta := \bar u(a_0) - \bar v(a_0)$ of weight from $a_0$ to its
    successor $a_1 \in A$, and call this transformed function $\bar u'$. Then
    \[
        r(w, \bar u') = r(w, \bar u) - \delta w(a_0) - \delta w(a_1) \leq r(w, \bar u).
    \]
    Since $\bar v$ can be obtained from $\bar u$ by a sequence of such transformations,
    the inequality follows.
\end{proof}
\begin{lemma}
    If $(u, v)$ is strongly fixed, then $u \b v$.
\end{lemma}
\begin{proof}
    Suppose by contradiction that $v < x_0 < \cdots < x_k < u$ is part of an optimal
    solution. Write $X = \sum_i x_i$ for the combined function. Then by assumption $r(X, u)
    \leq 0$, and therefore $r(X, v) = W_v r(X, \bar v) \leq W_v r(X, \bar u) = W_v / W_u r(X,
    u) \leq 0$. But then $c(X, u, v) < c(X, v, u)\leq c(v, X, u)$, which
    contradicts $(v,X,u)$ being optimal.
\end{proof}

\begin{remark}
    Consider $u \neq v$ from the original, unweighted problem, taking only values $0, 1$. Let 
    $n = |N(u)|, m = |N(v)|$, and consider both as ordered lists. Then $u, v$ are strongly
    fixed if and only if for all $0 \leq i < n$,
    \[
        N(u)_i \leq N(v)_{\lfloor i\cdot m/n \rfloor},
    \]
    or equivalently for all $0 \leq j < m$, $N(u)_{\lceil j\cdot n/m \rceil} \leq N(v)_j$.
\end{remark}

\begin{remark}
    Suppose that $u \neq v$ are not strongly fixed, and let $a_0 \in A$ be some element such
    that $\sum_{a \leq a_0} \bar u(a) > \sum_{a \leq a_0} \bar v(a)$, and $a_1 \in A$ its
    successor. By taking $X: A \to \mathbb R^{\geq 0}$ some function which assigns very high
    weight to $a_0$ and $a_1$, we can obtain $c(v, X, u) < \min(c(X, u, v), c(u, v, X))$, showing that
    \cref{lem:sfp} is optimal.
\end{remark}
Although such a function $X$ may exist in theory, it does
not have to exist in the actual set $B$, motivating the following definition.

\begin{lemma}[Practically fixed pair]\label{pfp}
  Suppose $r(u,v)\leq 0$.
    A \emph{blocking set} $X\subseteq B-\{u,v\}$ is a set such that $c(v,X,u) < \min(c(v, u,
    X), (X, v, u))$.  If there is no blocking set for $(u, v)$, then $u\b v$.
\end{lemma}

In practice, such a set $X$ can be found, if one exists, using a knapsack-like algorithm: for
each $x\in B-\{u,v\}$, add a point $P_x = (r(v, x), r(x, u))$, and search for a subset summing
to ${\leq{}(r(u, v), r(u, v))}$.

Note that we do not require $(v, X, u)$ to be a true local minimum, since we do
not consider interactions between vertices in $X$, as that would make ruling out the
existence of such sets much harder.

\subparagraph{Gluing}
We now turn our attention to \emph{gluing}, i.e., proving that two vertices $u$
and $v$ always go right next to each other. We start with a simple case of
\emph{gluing to the front}.

\begin{lemma}[Greedy]\label{greedy}
  When $r(u, x)\leq 0$ for all $x\in B$, there is a solution that
  starts with $u$.
\end{lemma}

\begin{remark}
  Let $u$ and $v$ satisfy $r(u,v) \leq 0$.
  When $N(u)=N(v)$ or more generally $\w_u=\w_v$, we can glue $u$ and $v$: $u\g v$.
  Otherwise, there is a multiset $\mathcal X\subseteq A$ such that $(u, X, v)$ is
  better than $(u,v,X)$ and $(X,u,v)$.
\end{remark}
This means that there is no `strong gluing'.
\begin{lemma}[Practical gluing]\label{pg}
  Let $u$ and $v$ satisfy $r(u,v)\leq 0$ and $u\be v$.
  A subset $X\subseteq B-\{u,v\}$ is \emph{blocking} when $c(u, X, v)$ is strictly
  better than $c(u,v,X)$ and $c(X,u,v)$.
  If there is no blocking set, then $u \g v$.
\end{lemma}
Again such sets $X$ can be found or proven to not exist using a knapsack
algorithm: add points $P_x = (r(u, x), r(x, v))$ and search for a non-empty
set summing to $\leq{}(0,0)$.

\begin{remark}[Tail variants]
  Our branch-and-bound method fixes vertices of the
solution from left to right. That means that at each step \cref{pg,pfp} can be applied to
just the \emph{tail}.
\end{remark}

\subsection{Branch-and-bound}
Our solver \texttt{OCMu64} is based on a standard branch-and-bound on the order the solution.
We start with fixed prefix $P=()$ and tail $T=B$, and in each step we try (a
subset of) all vertices in $T$ as the next vertex appended to $P$.
In a preprocessing step we compute the trivial lower bound $S_0 =
\sum_{u,v}\min(c(u,v),c(v,u))$ on the score.
We keep track of the score $S_P$ of the prefix and $S_{PT}=c(P, T)$ of
prefix-tail intersections, and abort when this score goes above the best
solution found so far. The \emph{excess} of a tail is its optimal score above
the trivial lower bound. We do a number of optimizations.

\begin{description}
  \item[Graph simplification] We drop degree-$0$ vertices, merge identical
    vertices, and split the graph into \emph{independent} components when
    possible. We find an initial solution using local search that tries to move
    slices and optimally insert them, and re-label all nodes accordingly to make
    memory accesses more efficient.
  \item[Dominating pairs] We find all dominating pairs and store them. For the
    exact track we also find practical dominating pairs. Instances for the parameterized
    track are simple enough that the overhead was not worth it. Also for each
    tail we search for new `tail-local' practical dominating pairs. In each state, we only
    try vertices $u\in T$ not dominated by another $v\in T$.
  \item[Gluing] We use the greedy strategy \cref{greedy}. Our implementation
    of \cref{pg} contained a bug, so we did not use this. (Also benefits
    seemed limited.)
  \item[Tail cache] In each step, we search for the longest suffix of $T$ that
    has been seen before, and reuse (the lower bound on) its excess. We also
    cache the tail-local practical dominating pairs.
  \item[Optimal insert] Instead of simply appending $u$ to the end of $P$, we
    insert it in the optimal position.\footnote{The idea is simple, but the
    implementation is tricky because it interacts in complicated ways with
    the caching of results for each tail.}
\end{description}

\section{Discussion}
\texttt{OCMu64} is a simple branch-and-bound method for solving one-sided
crossing minimization that does not depend on advanced ILP or SAT solvers.

This also seems to be its biggest limitation: while LP methods can use
the dual-space, \texttt{OCMu64} only considers the primal space to find lower
bounds on subproblems.

\end{document}
